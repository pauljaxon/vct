\documentclass[12pt,fleqn]{article}
%\usepackage[leftbars,dvips]{changebar}
%\usepackage[leftbars,pdflatex]{changebar}

\title{Victor: a SPARK VC Translator and Prover Driver \\[4ex]
  \Large
  User Manual for release 0.9.1 \\
  and subsequent experimental modifications \\[1ex]
% \normalsize
% \cbstart
%   \emph{Changebars indicate changes from User Manual for release 0.6 }\\[1ex]
% \cbend
%   \hbox{ }
} 

\author{Paul Jackson \\
        \texttt{pbj@inf.ed.ac.uk}}

\date{16 Mar 2012}

%\usepackage{vpage}
%\usepackage{vmargin}
%\setpapersize{Afour}

% margins l t r b, headheight, headsep, footheight, footskip
%\setmarginsrb{1.2in}{1.2in}{1.2in}{0.6in}{0.3in}{0.2inm}{0.3in}{0.3in}

\usepackage[a4paper,DIV14]{typearea} % LaTeX Companion, 2nd Ed, pp203-6
\usepackage{url}

\usepackage{array}

\setlength{\parindent}{0pt}

%My macros
%----------------------------------------------------------------------------
% Abbreviations
%----------------------------------------------------------------------------
\usepackage{amssymb}

\newcommand{\integer}{\ensuremath{\mathbf{Z}}}
\newcommand{\subrange}[2]{\{{#1}\:..\:{#2}\}}

\newcommand{\fixme}[1]{\textbf{FIXME: {#1}}}

\newcommand{\spark}{\textsc{Spark}}
\newcommand{\sparkb}{S{\footnotesize{}PARK}}  % for use in bold environments.
\newcommand{\ada}{Ada}
\newcommand{\sparkada}{SPARK-Ada}

\newcommand{\cvcthree}{\textsc{Cvc}3}
\newcommand{\zthree}{\textsc{Z}3}
\newcommand{\yices}{Yices}
\newcommand{\altergo}{Alt-Ergo}

\newcommand{\blast}{\textsc{Blast}}
\newcommand{\slam}{\textsc{Slam}}

\newcommand{\sat}{\textsc{Sat}}
\newcommand{\smt}{\textsc{Smt}}
\newcommand{\smtb}{S{\footnotesize{}MT}}
\newcommand{\sal}{\textsc{Sal}}
\newcommand{\etb}{\textsc{Etb}}

\newcommand{\sri}{\textsc{Sri}}
\newcommand{\smtlib}{\textsc{smt-lib}}

\newcommand{\fdl}{\textsc{fdl}}
\newcommand{\rls}{\textsc{rls}}
\newcommand{\vcg}{\textsc{vcg}}
\newcommand{\siv}{\textsc{siv}}

%NB the extensions here changed in V0.9, for SPARK tool compatibility
\newcommand{\logfile}{\textsc{vlg}}
\newcommand{\csv}{\textsc{vct}}
\newcommand{\sumfile}{\textsc{vsm}}


%\renewcommand{\And}{\ensuremath{\wedge}}
\newcommand{\And}{\ensuremath{\wedge}}
\newcommand{\Not}{\ensuremath{\neg}}
\newcommand{\Or}{\ensuremath{\vee}}
\newcommand{\Iff}{\ensuremath{\Leftrightarrow}}
\newcommand{\Implies}{\ensuremath{\Rightarrow}}

\newcommand{\tttilde}{{\tt \char`\~}}


% Boolean option with short one-line description
\newcommand{\optionbs}[1]{\item[\texttt{-{#1}}]}
% Boolean option with long description
\newcommand{\optionb}[1]{\item[\texttt{-{#1}}]\ \\}
% Value option
\newcommand{\optionv}[2]{\item[\texttt{-{#1}=}\mdseries\textit{#2}]\ \\}


%----------------------------------------------------------------------------
% Draft support
%----------------------------------------------------------------------------
\usepackage{comment}

\specialcomment{outline}%
  {\begingroup\bfseries\slshape\begin{itemize}}%
  {\end{itemize}\endgroup}


\specialcomment{question}%
  {\begin{itemize}\item[\Huge \textbf{?}]\itshape}%
  {\end{itemize}}

\specialcomment{remark}%
  {\begin{itemize}\item[\Huge \textbf{!}]\itshape}%
  {\end{itemize}}

% \excludecomment{outline}
% \excludecomment{question}
% \excludecomment{remark}

\newcommand{\todo}[1]{\textbf{ToDo:} \emph{#1}}
%\newcommand{\todo}[1]{\emph{\bfseries{#1}}}

\newenvironment{todoenv}{\begingroup\itshape}{\endgroup}



\usepackage[colorlinks=true]{hyperref}

\begin{document}

\maketitle

%\begin{abstract}
%This is a user guide for V0.8.
%\end{abstract}

\tableofcontents


\setlength{\parskip}{0.8\baselineskip}

%============================================================================
%\section{Introduction}
%============================================================================

%============================================================================
\section{Supported provers and prover languages}
%============================================================================

Victor has API interfaces to the \cvcthree{} and \yices{} provers, and
can drive any prover that accepts Simplify, \smtlib{} v1.2 or
\smtlib{} v2 format input files.

%----------------------------------------------------------------------------
\subsection{Simplify language}
%----------------------------------------------------------------------------

The Simplify language is supported by the Simplify and \zthree{} provers.

%----------------------------------------------------------------------------
\subsection{SMT-LIB language}
%----------------------------------------------------------------------------

The \smtlib{} initiative (\url{http://www.smtlib.org}) defines
standard languages for formatting input to \smt{} solvers and collects
benchmarks in this format. Victor supports the current version 2.0
and the older version 1.2.

The initiative defines particular \emph{logics}, defined by
combinations of logical theories and allowed shapes of terms and
formulas.  These logics are used to categorise benchmarks, and
different \smt{} solvers support different logics.  Some solvers 
support a superset of what is defined by any of these logics.

At a minimum, Victor requires solvers to handle uninterpreted
functions and constants, linear integer arithmetic and logical
quantifiers.

The following logics in both v1.2 and v2.0 provide at least this support:
\begin{itemize}
\item \textsc{auflia}: quantifier formulas involving arrays,
  uninterpreted functions, linear integer arithmetic.
\item \textsc{auflira}: quantifier formulas involving arrays,
  uninterpreted functions, linear integer and linear real arithmetic.
\item \textsc{aufnira}: quantifier formulas involving arrays,
  uninterpreted functions, non-linear integer and non-linear real
  arithmetic.
\item 
\textsc{ufnia}: Non-linear integer arithmetic with uninterpreted
  sort, function, and predicate symbols.
\end{itemize}
Victor can also take advantage of support for real arithmetic and for
sort symbols.  Such provers include Alt-Ergo, \cvcthree, \yices{} and
\zthree.
% Victor currently makes no use of the support for arrays
% provided by the above logics or by prover-specific
% extensions.

One major benefit of v2.0 is better support for queries involving
mixed real and integer arithmetic.  Victor does translate real
arithmetic to v1.2 of the \smtlib{} format. However, v1.2 does not
support well goals in which integers and reals are mixed. (For
example, it does not define a function injecting the integers into the
reals.)
  

%----------------------------------------------------------------------------
\subsection{Alt-Ergo}
%----------------------------------------------------------------------------

Alt-Ergo is an open-source \smt{} solver from LRI (Laboratoire de
Recherche en Informatique) at Universit\'e Paris-Sud.  It is available
from \url{http://alt-ergo.lri.fr/}.

Victor has been tested most recently with the V0.94 release using
Alt-Ergo's \smtlib{} 1.2 and 2.0 file-level interfaces.  Alt-Ergo's
1.2 interface works fine, but issues have been noted with the 2.0
interface:
\begin{enumerate}
\item \texttt{define-type} not supported.
\item \texttt{to\_real} function not recognised.
\item Instances of quantified Bool-typed variables are 
  not recognised.  Alt-Ergo 0.94 does not provide complete support
  for the merging of the syntactic category of formulas with the Bool
  type of terms.
  Sometimes multiple \texttt{unsat} responses are returned to a
  single \texttt{(check-sat)}.
\end{enumerate}
Victor has work-arounds for 1,2 and 4, but not 3.


% If the standalone Alt-Ergo executable is downloaded rather than built
% from the Alt-Ergo source distribution, it is also necessary to copy
% the file \texttt{smt\_prelude.mlw} from the source distribution to the
% \texttt{run/} directory.


%----------------------------------------------------------------------------
\subsection{CVC3}
%----------------------------------------------------------------------------

\cvcthree{} is an open-source \smt{} solver jointly developed at New
York University and the University of Iowa.  It is available from
\url{http://www.cs.nyu.edu/acsys/cvc3/}.

Victor can link to a \cvcthree{} library and can then drive
\cvcthree{} via its API.  Alternatively Victor can invoke a
\cvcthree{} stand-alone executable on \smtlib{} format files.

Victor has been tested with the latest release, V2.2, dating from
November 2009.  
%
% The \cvcthree{} developers also make available daily
% development releases.  Victor has not yet been tested with these.
%
\cvcthree{} is significantly slower than \yices{} or \zthree{} (maybe
5-10$\times$), especially when VCs are unprovable. 
%
It has some basic support for non-linear arithmetic.

When driven via its API, this version of \cvcthree{} throws exceptions
and has some segmentation faults on a few of the \spark{} VCs from the
tokeneer set,
%
The exceptions are caught and reported by Victor, but 
%
the segmentation faults cause Victor to halt.  To enable Victor runs
in the presence of these faulting problems, it is possible to tell
Victor to ignore trying to run \cvcthree{} on certain VCs.  


%----------------------------------------------------------------------------
\subsection{Simplify}
%----------------------------------------------------------------------------

Simplify is a legacy prover, used most notably in the ESC/Java tool. 

The Modula-3 sources and some documentation are available from HP labs.
Visit \url{http://www.hpl.hp.com/downloads/crl/jtk/index.html} and follow
the ``Download Simplify here'' link.

Executables for Linux and other platforms can be pulled out of the
ESC/Java2 distribution: visit
\url{http://secure.ucd.ie/products/opensource/ESCJava2/}.
In October 2007, the executables for V1.5.4 were found in a file 
\texttt{Simplify-1.5.5-13-06-07-binary.zip}.  These are 32-bit executables.
No success has been found in locating any 64-bit executables. 

Simplify has good performance, but is unsound and sometimes crashes
because it uses fixed-precision integer arithmetic.

Victor interfaces to Simplify using temporary files and by invoking
the Simplify executable in a sub-process.  Unlike the case with
\cvcthree{}, Victor can tolerate Simplify crashing.  Victor provides
notifications of Simplify crashes in its output files.


%----------------------------------------------------------------------------
\subsection{Yices}
%----------------------------------------------------------------------------

\yices{} is a state-of-the-art \smt{} solver available from \sri{} at
\url{http://yices.csl.sri.com/}.

Victor links with a \yices{} library provided with the \yices{}
distribution.  Victor has been tested with the latest public release,
V1.0.24.  This version, bug fixes apart, dates from summer 2007 and
essentially is the version that lead the field in the 2007 \smt{}
competition. 

% \yices{} 1 is no longer under development: \sri{} is
% currently working on a re-implementation, \yices{} 2, which had a
% preliminary showing at the 2008 \smt{} competition.  \yices{} 2 is not
% currently available, and 

\yices{} is fussy about VCs containing non-linear arithmetic
expressions.  Victor currently just has \yices{} ignore any hypotheses
or conclusions containing such expressions, and, not infrequently, VCs
are provable despite these ignored VC clauses.  
%

\yices{} will accept universally-quantified hypotheses with non-linear
arithmetic expressions, and sometimes can make use of linear
instantiations of these.
%
Unfortunately, the current behaviour on finding a non-linear
instantiation is abandon the proof attempt rather than simply ignore
the instantiation.  

No crashes have been observed with recent versions of \yices.
However, on a few VCs (not in the test sets provided with the
distribution), \yices{} just keeps going on and on.  No mechanism for
timing out on such cases has yet been implemented, the only way to
deal with them is to request that Victor ignore them.

Victor can also drive Yices using \smtlib{} format files. 

%----------------------------------------------------------------------------
\subsection{Z3}
%----------------------------------------------------------------------------

\zthree{} is a state-of-the-art \smt{} solver developed at Microsoft. See
\url{http://research.microsoft.com/en-us/um/redmond/projects/z3/}.

Victor has been tested most recently with the Linux version of release
3.2.  No problems have been observed with this version.

% This website only provides a Windows version of a Z3 executable.
% Microsoft research staff have verbally advertised the existence of a
% Linux version.  Victor has been tested with such a version (V1.3),
% obtained from Leonardo de Moura, \texttt{leonardo@microsoft.com}, one
% of \zthree's developers.

\zthree{} has good performance and better VC coverage than other
solvers tried.  In particular, it has the best support for non-linear
arithmetic.

Victor interfaces to \zthree{} using temporary files and by invoking
the \zthree{} executable in a sub-process.  The temporary files can be
in \smtlib{} 1.2, \smtlib{} 2.0 or Simplify format.

% Despite the similarity of the format, \zthree's performance can be radically
% different, depending on the format: it seems that \zthree{} has a soft
% timeout of under 0.1s when taking Simplify format input, but no such timeout
% on \smtlib{} format input.

%============================================================================
\section{Installation and Testing}
%============================================================================

Victor is written in C++ and currently only runs on Linux.  The
current distribution includes some preliminary code to allow it to
compile and run on Windows.  However, this code has not yet been fully
tested.

At Edinburgh, Victor is currently compiled and run on a Scientific Linux 6
64-bit platform.  It makes use of the following tools:
\begin{itemize}
\item \texttt{make} V3.81
\item \texttt{gcc/g++} V4.4.5
\item \texttt{bison}  V2.4.1
\item \texttt{flex} V2.5.35
\end{itemize}
% For SL5:
% \begin{itemize}
% \item \texttt{make} V3.81
% \item \texttt{gcc/g++} V4.1.2
% \item \texttt{bison}  V2.3
% \item \texttt{flex} V2.5.4
% \end{itemize}
The main external library it uses is
\begin{itemize}
\item \texttt{gmp} V4.3.1
%\item \texttt{gmp} V4.1.4
\end{itemize}
Its precise dependencies on these versions are largely unknown. 
%
One observation is that some tweaks to the \texttt{bison code} in
\texttt{parser.yy} were necessary when shifting from \texttt{bison}
V2.3 to \texttt{bison} V2.4.  Comments in \texttt{parser.yy} indicate
what needs to be changed for compilation with V2.3

By default, the \texttt{gmp} library is dynamically linked in.  If
running a single executable on several different Linux platforms, this
can cause problems and it might be desirable to use static linking
instead.  To achieve this, use \texttt{STATIC\_GMP=true} on the
\texttt{make} command line when building Victor.

To install:
\begin{enumerate}
\item Untar the distribution.  E.g. 
\begin{verbatim}
  tar xzf vct-0.9.0.tgz
\end{verbatim}
  This should generate a top level directory
  \texttt{vct-0.9.0} including subdirectories 
    \texttt{src},
    \texttt{bin},
    \texttt{run},
    \texttt{vc}
   and \texttt{doc}.
   The \texttt{doc} directory includes a copy of this manual.
   Other directories are described below.
\item 
  Configure Victor for each of the provers you wish to use it with.
  \begin{description}

  \item[\cvcthree:] To enable the API driver, uncomment the definition
    of variable \texttt{CVC3DIR} in file \texttt{src/Makefile} and edit
    its value to be that of your \cvcthree{} installation.  
    
    To use the \smtlib{} format file interface, ensure that an
    executable \texttt{cvc3} is on your current path.

  \item[Simplify:] Ensure an executable called \texttt{simplify} is on your
    current path. 

  \item[\yices:] 
    To enable the API driver, uncomment the definition of variable
    \texttt{YICESDIR} in file \texttt{src/Makefile}, and edit its
    value to be that of your \yices{} installation.

    To use the \smtlib{} format file interface, ensure that an
    executable \texttt{yices} is on your current path.

  \item[\zthree:]  Ensure an executable called \texttt{z3} is on your
    current path.

  \item[\altergo:]  Ensure an executable called \texttt{alt-ergo} is on your
    current path.
  \end{description}
  Alternate names, and optional paths can be specified for each executable
  at the top of the \texttt{Makefile} in the \texttt{run} directory.

  Victor can be run without driving any prover.  This is useful for
  testing if Victor's parser can handle certain VCs and for gathering
  information on VCs.  This mode can be used for compiling reports on
  the coverage obtained with the Simplifier prover provided with
  Praxis's \spark{} toolkit.

\item Build a Victor executable by \texttt{cd}ing to the \texttt{src}
  directory and typing \texttt{make}.  This does a variety of things, including
  \begin{enumerate}
  \item Creating \texttt{.d} files recording \texttt{make} rules that
     capture dependencies between source files.
   \item Running the \texttt{bison} parser generator and the \texttt{flex}
     lexer generator.
   \item Compiling various \texttt{.o} files.
   \item Linking the \texttt{.o} files together, along with prover
     libraries and the \texttt{gmp} library, and installing the
     resulting executable named \texttt{vct} in the \texttt{bin}
     directory.
  \end{enumerate}
  For convenience, a sub-directory \texttt{build} contains copies
  of files created during a build of Victor where it was configured
  for running with the Simplify and \zthree{} provers.  For example, if you
  do not have the correct version of \texttt{bison}, you could copy over
  the \texttt{bison} output files to the \texttt{src} directory.


\item Add the \texttt{vct/bin} directory to your \texttt{PATH}.
%  if you
%   wish to run Victor without specifying the path to this \texttt{bin}
%   directory.


\item Build utility tools for analysing the \csv{} and \logfile{}
  comma-separated-value
  output files created by Victor.  Enter \texttt{make csvutils}.  
  This causes the executables 
  \texttt{csvproj}, 
  \texttt{csvfilt},
  \texttt{csvmerge} and
  \texttt{csvisect} to be added to the \texttt{bin} directory.

\item Build an auxiliary program for timing-out runs of provers.
  Enter \texttt{make watchdog}.  This creates the executable
  \texttt{run/watchdogrun}.

\item Try running Victor on the example VCs provided with the distribution.
  Check the output files match the provided output files.  See Section
  \ref{sec:examples}
  for details.
\end{enumerate}

%============================================================================
\section{Operation}
%============================================================================
%----------------------------------------------------------------------------
\subsection{Terminology}
%----------------------------------------------------------------------------
We refer to \spark{} VCs as \emph{goals} and use the term \emph{goal
  slice} to refer to a proof obligation build from a VC by considering
just one of the goal's conclusions and ignoring the others.  

A \emph{unit name} is the hierarchical name of a program unit.  The unit
name with a \texttt{.fdl}, 
\texttt{.rls}, 
\texttt{.vcg} or 
\texttt{.siv} suffix gives a pathname for the corresponding
VC file relative to the root directory of all the VC files for a \spark{}
program.

%----------------------------------------------------------------------------
\subsection{Basic operation}
%----------------------------------------------------------------------------


The basic operation of Victor is to 
\begin{enumerate}
\item Read in a list of names of \spark{} program units.
\item Read in the VCs described in the \texttt{.fdl}, \texttt{.rls},
  \texttt{.vcg} file triples output by the \spark{} Examiner for the
  named program units.\footnote{
  Optionally it can read in the simplified
  \texttt{.siv} files output by the \spark{} Simplifier instead of the 
  \texttt{.vcg} files.}
\item Invoke a prover on each goal or goal slice. 
\item 
Output \texttt{.vct}, \texttt{.vsm} and \texttt{.vlg} report files.
\end{enumerate}

%----------------------------------------------------------------------------
\subsection{Input and output files}
%----------------------------------------------------------------------------
The Victor-specific input and output files are as follows.

% - - - - - - - - - - - - - - - - - - - - - - - - - - - - - - - - - - - - - -
\subsubsection{Unit listing input file}
% - - - - - - - - - - - - - - - - - - - - - - - - - - - - - - - - - - - - - -
\label{sec:unit-listing-file}

Typically Victor is run on many program units at once.  An input
\emph{unit listing} \texttt{.lis} file is used to indicate the units
it should consider.  
%
The grammar for each line in a unit listing file is given by

\noindent
\begin{tabular}{lll}
\textit{line}  & ::= & \textit{unitname} \{\textit{option}\}  \\[2ex]
%
\textit{option} & ::= & [\textit{tag}\texttt{?}]\textit{val} \\[2ex]
%
\textit{val} & ::= &  \textit{goal} \\
    & $|$  &  \textit{goal}\texttt{.}\textit{concl} \\
    & $|$  &  \textit{filename}\texttt{.fdl} \\
    & $|$  &  \textit{filename}\texttt{.rul} \\
    & $|$  &  \textit{filename}\texttt{.rlu} 
\end{tabular}

\noindent
where square braces ([]) enclose optional non-terminals,
curly braces (\{\}) enclose non-terminals repeated 0 or more times,
%
the terminals \textit{unitname}, \textit{tag} and \textit{filename} are
alphanumeric strings, 
and the terminals \textit{goal} and \textit{concl} are natural numbers.
%
The meaning of the components of a line are as follows.
\begin{itemize}
\item \textit{unitname} is the hierarchical name of a unit (\spark{}
  subprogram).

\item \textit{tag} tags an option.  A tagged option is only active if
  the tag is also supplied as one of the values of the 
  \texttt{-active-unit-tags} Victor command-line option.
  Untagged options are always active.

\item \textit{goal} and \textit{goal}\texttt{.}\textit{concl} 
  select particular goals and goal slices in the unit. 
  The Victor command-line options 
  \texttt{-include-selected-goals} and
  \texttt{-exclude-selected-goals}
  control how Victor treats these selected goals and goal slices.

\item \textit{filename}\texttt{.rul} and
      \textit{filename}\texttt{.rlu} are auxiliary rules files to load

\item \textit{filename}\texttt{.fdl} is an auxiliary declarations file.
  For example, this can declare constants and functions introduced in 
  an auxiliary rules file.
\end{itemize}
Comment lines are allowed: these are indicated by a \texttt{\#}
character in the first column.  Also blank lines are allowed.  

One way to prepare a unit listing file is to run the command
\begin{verbatim}
  find . -name '*.fdl' | sed -r 's/\.\/|\.fdl//g' > units.lis
\end{verbatim}
in the root directory of a set of VC files.


% - - - - - - - - - - - - - - - - - - - - - - - - - - - - - - - - - - - - - -
\subsubsection{VCT output file}
% - - - - - - - - - - - - - - - - - - - - - - - - - - - - - - - - - - - - - -

The \csv{} file includes one line for each goal or goal slice. The fields
of each line are:
\begin{enumerate}
\item Path to unit. This describes the containing packages
\item Unit name, without a prefix for the containing packages.
\item Unit kind.  One of \texttt{procedure}, 
  \texttt{function} or \texttt{task\_type}.
\item Source of path in subprogram for VC 
\item Destination of path for VC, or VC kind if not path related
\item VC goal number
\item Conclusion (goal slice) number 
\item Status (one of true, unproven, error)
\item Proof time (in sec)
\item Brief remarks about goal and solver interactions
\item Operator kinds occurring in hypotheses
\item Operator kinds occurring in conclusion
\end{enumerate}

% - - - - - - - - - - - - - - - - - - - - - - - - - - - - - - - - - - - - - -
\subsubsection{VSM output file}
% - - - - - - - - - - - - - - - - - - - - - - - - - - - - - - - - - - - - - -

The run summary file has extension \texttt{.vsm}.  It is a 1 line
comma-separated-value file with the fields
\begin{enumerate}
\item Report file name.
\item Number of \texttt{ERROR} messages in log file
\item Number of \texttt{WARNING} messages in log file
\item Total number of goal/goal slices processed.
\item Number of goal/goal slices with \emph{true} status
\item Number of goal/goal slices with \emph{unproven} status
\item Number of unproven goal/goal slices that involved a timeout.
\item Number of goal/goal slices with \emph{error} status
\item Percent of goal/goal slices with \emph{true} status
\item Percent of goal/goal slices with \emph{unproven} status
\item Percent of unproven goal/goal slices that involved a timeout.
\item Percent of goal/goal slices with \emph{error} status
\item Total execution time
\end{enumerate}
A file \texttt{vsm-file-header.txt} provides a 1 line comma-separated list
of headings for these summary files.

Summary files can be concatenated together with a header file and then 
viewed in any spreadsheet program.

% - - - - - - - - - - - - - - - - - - - - - - - - - - - - - - - - - - - - - -
\subsubsection{VLG output file}
% - - - - - - - - - - - - - - - - - - - - - - - - - - - - - - - - - - - - - -
The \logfile{} log file includes
\begin{enumerate}
\item a record of the command line options passed to Victor,
\item various information, warning and error messages,
\item statistics on the run, including numbers of VCs proven and unproven,
  and time taken.
\end{enumerate}

%----------------------------------------------------------------------------
\subsection{Invocation of Victor}
%----------------------------------------------------------------------------
The command line syntax for invoking Victor is
\begin{quote}
  \texttt{vct} [\textit{options}] [\textit{unitname}] 
\end{quote}
The \textit{unitname} argument is used to identify a single unit on which 
to run Victor.

To run Victor on multiple units, omit the \texttt{unitname} argument
and use instead the \texttt{units} option to specify a unit listing
input file.

If both a \textit{unitname} and a \texttt{units} option are provided,
the \texttt{units} option is ignored.

Victor takes numerous options, many of which are currently necessary.
See the next section for a description of a \texttt{Makefile} that
provides standard option sets.


%----------------------------------------------------------------------------
\subsection{Examples}
%----------------------------------------------------------------------------
\label{sec:examples}

The \texttt{vc} directory has subdirectories for some example 
sets of VCs.
%  that
% Victor has so far been tested on.
% \begingroup
% \renewcommand{\descriptionlabel}[1]{\texttt{#1}}
% \begin{description}
% \item[autop] The autopilot example distributed with the 
%   \spark{} book: ``High Integrity Software: The \spark{} Approach to 
%    Safety and Security'' by John Barnes.

% \item[hilton] Adrian Hilton's SPARK Missile Guidance Simulator.
% \item[tokeneer] The Tokeneer ID station case study prepared by Praxis
%   for the NSA.
% \end{description}
% \endgroup
See the file \texttt{vc/README.txt} for further information on these sets.
% sources of these VC sets.

The \texttt{run} directory provides a Makefile with rules for
running Victor on the VC sets in the \texttt{vc} directory.
%
These rules use Make patterns in their targets, and can easily also be
used for running Victor on users' own VC sets.
%
The rules set appropriate Victor command-line options and so allow
starting Victor users to ignore having to figure these out for
themselves.
%
See \texttt{run/Makefile} for details.

Reference report files obtained from running \texttt{make} on some of
these targets are included in directory \texttt{run/out-ref}.
%
Unix \texttt{diff} can be used to check that newly-generated
report files are the same as the reference files.
%
If the command line option \texttt{-gstime} is used to include times
of prover runs in report files, it will be necessary to use
the \texttt{csvproj} utility to remove the field for these times
in order to get files that are expected to be identical.

%----------------------------------------------------------------------------
\subsection{Performance tips}
%----------------------------------------------------------------------------

\begin{enumerate}

\item When \smt{} solvers cannot prove a goal, they often keep trying
  almost indefinitely rather than halting, so it is good to run them
  with some kind of time-out.  
  %
  When several VCs cannot be proven, Victor's total run-time can be
  dominated by the runs that go to time-out.  Setting a shorter
  time-out can therefore sometimes radically reduce Victor's run-time,
  often with little or no drop in number of goals proven.

\item \smt{} solver performance on goals they can prove is often
  dependent on the number of quantified axioms. 
  By default, Victor uses a number of quantified axioms from the 
  rules files \texttt{divmod.rul} and \texttt{prelude.rul} in the \texttt{run/}
  directory. 
  %
  In some cases, not all these axioms are necessary, and faster
  run-times are achievable with alternate rules files that prune down
  these axiom sets.

\end{enumerate}


%============================================================================
\section{Command line options}
%============================================================================
Options are specified with syntax 
\texttt{-\emph{name}} or
\texttt{-\emph{name}=\emph{value}}.
%
Option values can be boolean (\texttt{true} or \texttt{false}),
natural numbers (e.g. \texttt{42}) or strings.
%
An option \texttt{-\emph{name}} is interpreted the same as 
An option \texttt{-\emph{name}=true}.  An unset boolean option is interpreted
as \texttt{-\emph{name}=false}.

If the same option is given multiple times with different values, the
usual behaviour is that the last value is taken.  Occasionally all multiple 
values are used.  These cases are always explicitly pointed out below.

A later option 
\texttt{-name=},
\texttt{-name=false},
\texttt{-name=none} or 
\texttt{-name=default}
Clears all earlier values given for the option and makes it unset.

%----------------------------------------------------------------------------
\subsection{Input options}
%----------------------------------------------------------------------------

These options control where VCs are read from, provision of auxiliary
declarations and rules, and filtering of VCs before invoking the selected
prover.

\begin{description}

  \optionv{units}{unit-listing}  Run on units named in \textit{unit-listing}
   file.

  \optionv{prefix}{prefix} 
    Use \textit{prefix} as a common prefix for all unit names.  
    \textit{prefix}\texttt{/}\textit{unitname} should give an absolute or
    relative pathname for the VC file set of a program unit.  Default is
    that no prefix is used.
   
  \optionv{decls}{declfile}
    For every program unit, read auxiliary \texttt{.fdl} 
    declarations file named in \textit{declfile}.  
    Multiple files can be specified using multiple 
    \texttt{-decls} options.
%     For example, the Examiner does not provide declarations for built-in
%     bit operations such as \texttt{bit\_\_or} and \texttt{bit\_\_not}.

  \optionb{read-all-decl-files-in-dir}
  For each program unit with full name \textit{dirs/dir/unitname},
  read in all \fdl{} declaration files in directory \textit{dirs/dir/} rather
  than just the \fdl{} declaration file \textit{dirs/dir/unitname.fdl}.
  This option is useful in conjunction with 
  \texttt{-read-directory-rlu-files}
  and \texttt{-read-unit-rlu-files} when the user-defined rules files for a
  unit 
  (see options \texttt{-read-directory-rlu-files} 
   and \texttt{-read-unit-rlu-files})
  refer to identifiers that are not declared in \textit{dirs/dir/unitname.fdl}.
  An alternative option is \texttt{-expect-dir-user-rules-with-undeclared-ids}.

  \optionb{read-directory-rlu-files}
  For each program unit with full name \textit{dirs/dir/unitname},
  read in the user-defined rules file \textit{dirs/dir/dir.rlu}.  The SPARK
  Simplifier reads in such rules files by default.

  \optionb{read-unit-rlu-files}
  For each program unit with full name \textit{dirs/dir/unitname},
  read in the user-defined rules file \textit{dirs/dir/unitname.rlu}.  
  The SPARK Simplifier reads in such rules files by default.

  \optionv{rules}{rulesfile} For every program unit, read in
    auxiliary rules file named in \textit{rulesfile}.  Multiple files
    can be specified using multiple \texttt{-rules} options.

  \optionb{expect-dir-user-rules-with-undeclared-ids}
  Directory-level user-defined rules files can sometimes include rules
  that involve type, function and constant identifiers that are not
  declared in the \fdl{} file for a unit being processed.  Normally, when
  Victor encounters a rule that has some undeclared identifiers, it
  deletes that rule and outputs a warning message to the log file.
  With this option, an info message rather than a warning message is
  output for directory-level user rules involving undeclared ids.  An
  alternative option is \texttt{read-all-decl-files-in-dir}.

  \optionb{warn-about-speculative-overload-resolution}
  Several \fdl{} operators are overloaded, and this overloading has to be
  resolved in Victor.  Some overload resolution tactics are more 
  speculative than others. Victor always logs warnings about applications
  of the more speculative tactics. With this option, it will also 
  log warnings rather than info messages about applications of the 
  less speculative tactics.

  \optionb{siv} Read in \texttt{.siv} simplified VC files output by the
     Simplifier rather than \texttt{.vcg} Examiner VC files.

% See smt-driver.cc
\optionv{goal}{g} Only consider goal number $g$.
  %  Skipped over goals are not reported in \csv{} output file.  
  %
  This option is intended for use when Victor is run on a single unit,
  when a \textit{unit-name} argument and no \texttt{units} option is
  given.

\optionv{concl}{c} Only consider conclusions (goal slices) numbered $c$.  
  This option is intended for use when Victor is run on a single unit.

\optionb{skip-concls}
  Do not pass conclusion formulae to the selected prover.  This option is
  good for helping to identify goals or goal slices true because of an
  inconsistency in the hypotheses or rules.

\optionb{skip-hyps}
  Do not pass hypothesis formulae to the selected prover.  This option is
  good for helping to identify goals or goal slices true because of an
  inconsistency in the rules.

% See utility.cc
\optionv{from-unit}{unit-name}
  Only drive to selected prover the units listed in the \texttt{-units} option
  starting with \textit{unit-name}.  Default is to start with first. 

\optionv{from-goal}{g}
  In first unit to be passed to prover, start driving goals / goal
  slices to prover at goal $g$.

\optionv{to-unit}{unit-name}
  Stop driving units to the selected prover after \textit{unit-name} is 
  encountered. Default is to continue until the last listed unit.

\optionv{active-unit-tags}{tags}
  Identify which tagged options (if any) in the unit listing file to make
  active.  See Section \ref{sec:unit-listing-file} for more on this.
  %
  Multiple tags should be separated by colons (\texttt{:}).

\optionb{include-selected-goals}
  When particular goals or goal slices are selected for a unit in the
  unit listing file, run Victor on just those goals or goal slices.

\optionb{exclude-selected-goals}
  When particular goals or goal slices are selected for a unit in the
  unit listing file, do not run Victor on those goals or goal slices.

\end{description}
%----------------------------------------------------------------------------
\subsection{Translation options}
%----------------------------------------------------------------------------

See Section \ref{sec:translation} for a presentation of these options,
since they make best sense in a discussion of the overall translation
process.

%----------------------------------------------------------------------------
\subsection{User rule analysis}
%----------------------------------------------------------------------------
\label{sec:user-rule-analysis}

These options analyse the rules that users add to the directory-level
and unit-level .rlu user rule files.  Users add rules to overcome
incompletenesses in the Simplifier prover and to provide extra
information that cannot easily be routed through the Examiner.  These
options help identify those rules required by the Simplifier, but not
some SMT solver.  They also help to pick out rules that inadvertently
are inconsistent.


%- - - - - - - - - - - - - - - - - - - - - - - - - - - - - - - - - - - - - - 
\subsubsection{Finding redundant rules}
%- - - - - - - - - - - - - - - - - - - - - - - - - - - - - - - - - - - - - - 

\begin{description}
  \optionb{find-redundant-rules} Consider each goal that is provable
  with all the user rules, and determine a minimal set of user rules
  needed to prove the goal.
    \begin{itemize}
    \item Report on which unit-level user rules are not in any of the
      minimal sets of the goals for a given unit.
      These unit-level user rules are redundant.
    \item Across all the goals in the units in a given directory, report 
      on which directory-level user rules are
      are not in any of the minimal sets for each goal in each unit.
    \end{itemize}
    The minimal set for a goal is found in a greedy fashion, starting
    with the full set of user rules, and then seeing if each rule in
    turn can be knocked out without affecting the provability of the
    goal.
    %
    The minimal set is similar to a proof summary / unsat core 
     see the documentation on the \texttt{-smtlib2-unsat-cores} option in 
    Section \ref{sec:prover-driving-options}).
    However, an unsat core might well include a user rule in a core because
    the SMT solver happened to first discover a proof using the rule, even if
    there are proofs without the rule.
    Here, when a user rule is considered, it is retained only if it is seen
    to be definitely needed.

\optionb{report-user-rule-status}
% was report-excluded-rlu-rules-per-goal
  For each goal, report the .vlg log file on the status of each user
  rule, whether it was
  \begin{enumerate}
  \item excluded from consideration because of unbound function,
    constant or type identifiers,
  \item considered and found to be outside the minimal set for that goal,
  \item considered and found to be inside the minimal set.
  \end{enumerate}

  
\end{description}
%- - - - - - - - - - - - - - - - - - - - - - - - - - - - - - - - - - - - - - 
\subsubsection{Auditing rules}
%- - - - - - - - - - - - - - - - - - - - - - - - - - - - - - - - - - - - - - 

In a \emph{rule audit}, for each unit Victor completely discards the
goals read in from .vcg or .siv files and instead generates a set of goals
that check various properties of the user rules.  
%
The newly generated goals are divided into 5 kinds, as shown in
Table~{\ref{tab:audit-goal-kinds}}.
\begin{table}[htb]
  \newcommand{\uiname}{$u_i$-name}
  \centering
  \begin{tabular}{ccp{6cm}>{\ttfamily\ \ }ll}
    Kind   &    Goal shape  &  Description  &  \multicolumn{2}{c}{.vct tag} \\
    \hline
    A      &  $R \vdash \bot$ 
                      &  Are system rules inconsistent?
                                      & \ttfamily no urules  \\
    B      &  $R, U \vdash \bot$
                      &  Are user rules inconsistent?
                                      & all urules  \\
    C      &  $R, u_i \vdash \bot$
                      &  Is user rule $u_i$ inconsistent?
                                      & urule as H     & \uiname \\
    D      &  $R, \vdash u_i$
                      &  Is user rule $u_i$ derivable with no use of
                         other user rules?
                                      & urule as C   & \uiname \\
    E      &  $R, U\setminus\{u_i\} \vdash u_i$
                      &  Is user rule $u_i$ derivable from 
                         other user rules?
                         & urule from rest   & \uiname\\
  \hline
  \end{tabular}
  \caption{Kinds of Rule Audit Goals}
  \label{tab:audit-goal-kinds}
\end{table}

$R$ is the set of system rules, either generated by the Examiner
and read in from an .rls file, read in by Victor from one of its
standard rules files (e.g. \texttt{prelude.rul}), or generated by
Victor as part of its translation, 
%
and $U = \{u_1, \ldots, u_n\}$ is the set of user rules read in from
directory-level and unit-level user rule files, excluding those directory-level
rules that are not well formed because they have unbound constant, function or type identifiers.
%
For each unit, 1 goal of each of kinds A and B is generated, and $n$
goals of each of kinds C,D and E are generated.
%
The two columns labelled \emph{.vct tag} show what is output to the
\emph{goal origins} fields in the goal-based .vct report file.

Victor writes the number of goals of each kind that found to be true
to the .vus unit summary file, the .vsm session summary file and in an
extra audit report at the end of the .vlg log file.

The options are as follows.
\begin{description}
\optionb{do-rule-audit} Generate audit goals and collect results 
  in report files.
\optionb{rule-audit-a}  Include goals of kind A in audit.
\optionb{rule-audit-b}  Include goals of kind B in audit.
\optionb{rule-audit-c}  Include goals of kind C in audit.
\optionb{rule-audit-d}  Include goals of kind D in audit.
\optionb{rule-audit-e}  Include goals of kind E in audit.
\optionv{rule-audit-rule}{rname} 
  Generate audit goals just for rule with name \emph{rname} rather than
  for all of $u_1, \ldots, u_n$.  This option makes sense when attention
  is focussed on one unit or, if a directory-level rule is named, all
  units in a single directory.
\end{description}

%----------------------------------------------------------------------------
\subsection{Prover and prover interface selection}
%----------------------------------------------------------------------------

\begin{description}
\optionv{prover}{prover}
  Select the prover to drive.  Valid values of \emph{prover} are:
  \begin{itemize}
  \item \texttt{cvc3}
  \item \texttt{simplify}
  \item \texttt{yices}
  \item \texttt{z3}
  \end{itemize}
  A value of \texttt{none} can also be specified.  This is useful if
  one just wants to generate prover input files.

\optionv{prover-command}{prover-command}
  Use instead of the \texttt{prover} option to specify explicitly a
  shell-level command for invoking the prover.  This allows alternate
  provers or custom prover options to be specified.

  Selecting neither this option or the \texttt{prover} option is
  equivalent to setting the value of \texttt{prover} to \texttt{none}.
  
\optionv{interface-mode}{mode}
  Select the prover interface mode.  Valid values of \emph{mode} are:
  \begin{itemize}
  \item \texttt{api}: Use prover API.  
    Acceptable with \texttt{cvc3} or \texttt{yices}
    value for \texttt{prover}.

  \item \texttt{smtlib}: Use \smtlib{} 1.2 format files and
     stand-alone prover executable.
    Acceptable with \texttt{cvc3}, \texttt{yices} or \texttt{z3}
    value for \texttt{prover},
    and with \texttt{prover-command} option.

  \item \texttt{smtlib2}: Use \smtlib{} 2.0 format files and
     stand-alone prover executable.
    Acceptable with \texttt{cvc3} or \texttt{z3}
    value for \texttt{prover},
    and with \texttt{prover-command} option.

  \item \texttt{simplify}: Use Simplify-format files and
     stand-alone prover executable. 
    Acceptable with \texttt{simplify} or \texttt{z3} value for
    \texttt{prover}, 
    and with \texttt{prover-command} option.

%   \item \texttt{alt-simplify}: Use alternate Simplify-format files and
%      stand-alone prover executable. 
%     Acceptable with \texttt{simplify} or \texttt{z3} value for
%     \texttt{prover}, 
%     and with \texttt{prover-command} option.
%     These alternate Simplify-format files are generated by a customisation
%     of the code generating \smtlib{} interface.
  \item \texttt{dummy}: 
    Use some default code that mostly does nothing.  In this case,
    Victor still parses the VC files, does a prover independent
    translation of the goals, and generates \csv{} and log output
    files.  This is the default option. 
  \end{itemize}

\end{description}

% \begingroup
% \newcommand{\tick}{$\bullet$}
% \begin{tabular}{l|ccccc}
%   Mode     &   \multicolumn{5}{c}{Prover}   \\
%            &  cvc3    & simplify   & yices & z3   & \textit{\rmfamily custom}\\
%   \hline

%   api      &  \tick   &            & \tick &      &          \\

%   simplify &          & \tick      &       &\tick & \tick         \\
%   smtlib   &  \tick   &            & \tick &\tick & \tick

  
% \end{tabular}
% \endgroup


%----------------------------------------------------------------------------
\subsection{Prover driving options}
%----------------------------------------------------------------------------
\label{sec:prover-driving-options}

\begin{description}
\optionb{fuse-concls} 
  Pass one goal at a time to the selected prover.  By default Victor
  passes one goal slice at a time.

\optionv{working-dir}{working-dir} 
  Use \textit{working-dir} as root of directory tree of files used for prover
  input and output.   An argument of  `\texttt{.}' is acceptable to indicate
  the current directory.  Defaults to \texttt{/tmp}.  
%  Option is currently only relevant for Simplify and \zthree.

  Unless one of the next three options is used, the same file
  names are used for every every prover run and every Victor run.

\optionb{hier-working-files}
   Use distinct files for each prover invocation and arrange in a
   hierarchical tree under \emph{working-dir}.

\optionb{flat-working-files}
   Use distinct files for each prover invocation and arrange all as 
   members of \emph{working-dir}.


\optionb{unique-working-files}
  Within a given Victor run, use the same file names for each prover
  invocation, but, by including hostname and process
  number in file names, make the names unique to the Victor run.
  %
  This option is useful if one wants to have simultaneous Victor runs.

\optionb{delete-working-files}
  Delete the files used for prover input and output after
  each prover invocation.

\optionb{add-formula-descriptions} 
  Add comments in the prover input file describing the rule, hypothesis
  or conclusion that each asserted formula comes from.

\optionb{smtlib2-unsat-cores}
  Ask the prover to report an unsatisfiability core for each proven goal.
  An \emph{unsatisfiability core} or \emph{unsat core} names which of
  the rules, hypotheses and the conclusion of a goal were used in the
  proof of the goal.  An unsat core can be viewed as a proof summary.
  Unsat cores are reported in the remarks field of records written to the
  \csv{} file.

  Unsat cores are currently only supported if the \smtlib{}2 file-level
  prover interface is used.

  Generation of unsat cores can slow down provers.  For example,
  \zthree's run-time on provable goals has been observed more than
  double when unsat core generation has been requested.

\optionv{ulimit-timeout}{time}
  If using either of the file-level interface modes, 
   use the Linux 
  \emph{ulimit} process limit facility to time out
  prover invocations after \emph{time} seconds.  The \texttt{time} value 
  should be a natural number.
  The default is not to time out prover invocations.

\optionv{watchdog-timeout}{time}
  %
  If using the \smtlib{}2 file-level interface mode, use the provided
  \emph{watchdogrun} C program to time out prover invocations after
  \emph{time} seconds of inactivity on the prover's output.  The
  \texttt{time} value can be a natural number or a fixed-point
  number (e.g. 0.1).  The default is not to time out prover invocations.

  The timeout measurement from the last output activity (or prover
  start time) is useful when a single call of the prover is used to
  answer multiple queries.  When timeouts are measured this way, they
  bound the run-time of the prover on the current query it is
  attempting, rather than also including in the allowed run period the
  times spent by the prover answering previous queries in a query set.

\optionv{smtlib2-soft-timeout}{time}
 %  
 If using the \smtlib{}2 file-level interface mode and \zthree{} v3.1 or newer,
 timeout \emph{check-sat} prover invocations after \emph{time} milliseconds.
 The \texttt{time} value should be a natural number.

\optionv{shell-timeout}{time}
  If using either of the file-level interface modes, 
   use the provided shell script 
   \texttt{timeout.sh} to time out prover invocations after \emph{time} 
  seconds.  The \texttt{time} value can
  be a natural number or a fixed-point number (e.g. 0.1).
  The default is not to time out prover invocations.

  Currently this option is not that robust and use of \texttt{-ulimit-timeout}
  is recommended instead.

\optionv{logic}{logic}
  If using the \smtlib{} interface mode, set the value of the 
  \texttt{:logic} attribute in the \smtlib{}-format files to \emph{logic}.
  The default is \texttt{AUFLIA}.

\optionb{smtlib-hyps-as-assums}
  If using the \smtlib{} interface mode, insert each hypothesis into
  the \smtlib{}-format file as the value of a distinct 
  \texttt{:assumption} attribute.
  
\optionb{use-alt-solver-driver}
  This option enables an alternative driver.  The new
  major features supported by this driver are \emph{incrementality} (see
  next option) and \emph{user rule analysis} (see
  Section~\ref{sec:user-rule-analysis}).
  
\optionb{exploit-solver-incrementality}
  This option gives significant reductions in prover run-time.

  Prior to summer 2011, the main solver driver always handed goals
  to the solver one at a time.  Each solver invocation had a complete
  separate copy of all the declarations and rules, even though these
  are constant across the goals of a unit. 

  With this option, when driving a unit, declarations and rules are
  passed to the solver just once, rather than repeatedly for each
  goal.  Use is made of solver support for pushing and popping
  contexts.

  This option has been tested with the \smtlib{} 2 file-level
  interface.  It ought to work with both the \yices{} and \cvcthree{}
  API interfaces, though this has not yet been fully enabled.

  When using the file-level \smtlib{} 2 interface, the driver
  initially creates a single \smtlib{} 2 format file containing
  queries for all the goals in a unit.  If the solver is killed
  because it timed out on some goal, a new solver input file is
  created just for the remaining goals. On each further timeout, a
  further new solver input file is created.

  This option is best used in conjunction with
  \texttt{-watchdog-timeout} or \texttt{smtlib2-soft-timeout} to
  ensure timeouts are applied to each individual query.  The
  \texttt{smtlib2-soft-timeout} option gives better performance, as
  the solver is then not killed on timeouts, and after a timeout can
  continue on further goals from the same input file: there is no need
  for rerunning the solver on a new file containing the remaining
  goals. However, for some reason, when run with the same time limits,
  the coverage with the soft timeout is lower.  This is currently being
  investigated.
  


% % smt-driver.cc
% \optionv{drive-goal-repeats}{count}
%    Repeat each prover invocation \textit{count} times.
%    This is used to increase precision of prover runtime measurements when
%    using an API interface.

% \optionv{check-goal-repeats}{count}
%    Repeat each prover invocation \textit{count} times.
%    This is used to increase precision of prover runtime measurements when
%    using a file-level interface.  

\end{description}

%----------------------------------------------------------------------------
\subsection{Output options}
%----------------------------------------------------------------------------


%- - - - - - - - - - - - - - - - - - - - - - - - - - - - - - - - - - - - - - 
\subsubsection{Screen output options}
%- - - - - - - - - - - - - - - - - - - - - - - - - - - - - - - - - - - - - - 

\begin{description}
\optionb{utick} Print to standard output a \texttt{*} character at the 
  start of processing each unit.
  If \texttt{-longtick} also selected, print instead 
  the unit name.
\optionb{gtick} Print to standard output a \texttt{;} character at the 
  start of processing each goal.  
  If \texttt{-longtick} also selected, print instead 
  the goal number.
\optionb{ctick} Print to standard output a \texttt{.} character at the 
  start of processing each conclusion of a goal.
  If \texttt{-longtick} also selected, print also
  the conclusion number.
\optionb{longtick} See above.

\optionb{echo-final-stats}
  Print to standard output the final statistics that are included at the
  end of the report file.
\end{description}


%- - - - - - - - - - - - - - - - - - - - - - - - - - - - - - - - - - - - - - 
\subsubsection{General report file options}
%- - - - - - - - - - - - - - - - - - - - - - - - - - - - - - - - - - - - - - 

\begin{description}
\optionv{report}{report-file}
   Use \textit{report-file} as body of filenames for \csv{}, \sumfile{} and
   \logfile{} report files.  Default is to use \texttt{report}.

\optionv{report-dir}{dir}
   Put report files in directory \textit{dir}.  If directory does not exist,
   it is created.  Default is to use current directory.
\end{description}

%- - - - - - - - - - - - - - - - - - - - - - - - - - - - - - - - - - - - - - 
\subsubsection{VCT file options}
%- - - - - - - - - - - - - - - - - - - - - - - - - - - - - - - - - - - - - - 


\begin{description}

\optionb{count-trivial-goals}
   Write an entry to the \csv{} file for each input goal of form 
   \texttt{*** true}.
   In VCG files, these are the goals proven by the Examiner.
   In SIV files, these are the goals proven by the Examiner or the 
   Simplifier.
   These entries have status \texttt{true} in field 8 and
   the comment \texttt{trivial goal} in field 10.

   Use this option along with option \texttt{-fuse-concls} to have the
   goal counts match those from the POGS (Proof Obligation Summariser)
   tool.

\optionb{hkinds} Report list of hypothesis kinds in field 11 of \csv{} file
\optionb{ckinds} Report list of concl kinds in field 12 of \csv{} file
\optionb{gstime} 
  Report time taken by prover to process a goal slice or goal
       in field 9 of \csv{} file.
\optionb{gstime-inc-setup}
  Include setup time in gstime.  This setup time is time to 
  send declarations, rules, hypotheses and conclusions to the prover before
  invoking prover itself.  

  It is appropriate to include this time when calling the prover via 
  an API (\yices and \cvcthree{} cases) since
  the provers do incremental processing on receiving this information.
  When the prover interface is via files, this setup time is the time to 
  write an input file for the prover, so it is not as appropriate to include
  it.

\optionb{csv-reports-include-goal-origins}
  Include information on goal origins in fields 4 and 5 of \csv{} file.
  Default is not to include this information.

\optionb{csv-reports-include-unit-kind}
  Include information on unit kind in fields 3 of \csv{} file.
  Default is not to include this information.
\end{description}

%- - - - - - - - - - - - - - - - - - - - - - - - - - - - - - - - - - - - - - 
\subsubsection{Log file options}
%- - - - - - - - - - - - - - - - - - - - - - - - - - - - - - - - - - - - - - 

\begin{description}
  \optionv{level}{level}  Report all messages at or above 
     priority \textit{level}.  The levels and associated names are
     \begin{enumerate}
     \item [6] \texttt{error}
     \item [5] \texttt{warning}
     \item [4] \texttt{info}
     \item [3] \texttt{fine}
     \item [2] \texttt{finer}
     \item [1] \texttt{finest}
     \end{enumerate}
     The \textit{level} value can either be a number or the associated name.
     The default level is \emph{warning}.

\end{description}

%----------------------------------------------------------------------------
\subsection{Debugging options}
%----------------------------------------------------------------------------
\begin{description}
\optionb{scantrace} Write lexer debugging information to standard output 
\optionb{parsetrace} 
  Write parser debugging information to standard output 
\end{description}

%----------------------------------------------------------------------------
\subsection{CVC3 options}
%----------------------------------------------------------------------------

Unless otherwise specified, these options are only relevant when invoking
\cvcthree{} via its API.
%
The main options are as follows.
\begin{description}
\optionb{counterex}
   Report counterexamples for false and unknown queries. 

\begin{remark}
I have not figured out yet how to direct \cvcthree{} to write counter-examples
to files.   A work-around to view counter-examples is to run with 
this option and the \texttt{-cvc-inputlog} option, and then run the
standalone \cvcthree{} executable on the generated \cvcthree{} input file.
\end{remark}

% \optionv{resourcelimit}{limit}
%   Set resource limit for \cvcthree{} runs.  This provides a soft timeout
%   option.
%   A value of 25,000 gives run time limits of 7-10sec on a 1.86GHz Intel
%   Core 2 processor.  Default is 0, no limit.

%   This option also works when invoking \cvcthree{} on \smtlib{}-format input
%   files.

\optionv{timeout}{time}
  Set a timeout period in units of 0.1 seconds for runs of \cvcthree{}, both
  via API and via executable.  Uses \cvcthree{}'s internal support for
  timing out.
  
\optionb{cvc-loginput}
  Enable echoing of API calls for each \cvcthree{} run to a file. 
  Use the \texttt{-working-dir} option to set where the file is stored
  and the 
  \texttt{-hier-working-files} and \texttt{-flat-working-files} options
  to control whether and how distinct files are used for each run. 
  %
  Files have suffix \texttt{.cvc}.
  %
  If distinct files are not requested,
  all runs will be echoed to a file named \texttt{cvc3.cvc},
  each run overwriting the previous one.
  %
  These files are saved in \cvcthree{}'s standard input language and can be 
  used as input to a \cvcthree{} stand-alone executable.

% \optionv{cvc-echo-suffix}{suffix}
%   Set suffix to use for file logging \cvcthree{} calls.  Defaults to 
%   \texttt{.clog}.
\end{description}
See the file \texttt{cvc-driver.cc} for further available options.  Not
all of these have been tried out yet.


%----------------------------------------------------------------------------
\subsection{Simplify options}
%----------------------------------------------------------------------------

No options are currently available.

%----------------------------------------------------------------------------
\subsection{Yices options}
%----------------------------------------------------------------------------

Unless otherwise specified, these options are only relevant when invoking
\yices{} via its API.

\begin{description}
\optionb{yices-loginput}
  Enable echoing of API calls for each \yices{} run to a file. 
  Use the \texttt{-working-dir} option to set where the file is stored
  and the 
  \texttt{-hier-working-files} and \texttt{-flat-working-files} options
  to control whether and how distinct files are used for each run. 
  %
  Files always have suffix \texttt{.yices}.  If distinct files are not
  requested, all runs will be echoed to the file 
  \texttt{yices.yices}, each run overwriting the previous one.
  %
  These files are saved in \yices{}'s standard input language and can be 
  used as input to a \yices{} stand-alone executable.
\begin{remark}
%   The \yices{} executable preloads declarations of functions \texttt{div}
%   and \texttt{mod}.  The \yices{} input files output by Victor might
%   need editing to remove declarations of these functions before the input
%   files will execute properly. 
  %
  Victor lets \yices{} reject non-linear parts of formulae - see the
  warnings in Victor's log file.  These formulae might have to be removed by 
  hand for \yices{} to load these input files properly.
\end{remark}


\optionb{yices-logoutput}
  Set file for output of each run of \yices{}. 
  Location of file and whether distinct files generated for each run
  are specified in same way as with \texttt{-yices-loginput}.
  Suffix of files is \texttt{.ylog}.  If distinct files not requested,
  all runs written to \texttt{yices.ylog}.

% \optionv{yverb}{n} 
%   Set verbosity of \yices's output to \textit{n}.

\optionb{counterex} 
   Enable reporting of counter-example models to output log file.

% -ynotc          Disable Yices type checking (enabled by default).

% -abstract-nonlin-divmod
%                 Abstract nonlinear divmod exps to uninterpreted functions.
%                 Default is to abstract all divmod exps.
% -abstract-nonlin-times
%                 Abstract nonlinear times exps to uninterpreted functions.
%                 Default is to abstract no times exps.

\optionv{timeout}{time}
  Set a timeout period in seconds for runs of the \yices{} executable 
  on \smtlib{}-format input files.   Uses \yices{}'s \texttt{--timeout}
  option.

\end{description}
%\fixme{Do anything about non-linear arithmetic options?}


%----------------------------------------------------------------------------
\subsection{Z3 options}
%----------------------------------------------------------------------------

No \zthree{} options are specifically supported by Victor.  \zthree{} options
can be specified by giving a custom prover command with the \texttt{prover-command} option.

% \begin{description}
% \optionb{z3-fourier-motzkin}
%   Use Fourier-Motzkin elimination to eliminate all quantifiers over linear
%   arithmetic expressions.
% \end{description}



%============================================================================
\section{Translation}
%============================================================================
\label{sec:translation}

% As defined in (p:) processor.cc and (t:) translation.cc

The description of the translation
process here is rather brief and not self-contained.
%
The process is best understood by first having a read of the
draft paper \emph{Proving SPARK Verification Conditions with SMT solvers},
available from the author's website.

Unless otherwise stated, translation steps are carried out in order
they are described in below.

%----------------------------------------------------------------------------
\subsection{Standard Form translation}
%----------------------------------------------------------------------------

Most translation steps in Victor are carried out on units in a standard form.
In this standard form all functions and relations have a unique type, 
there is no overloading.

The first translation step is to put units into this standard form.

% p;putUnitInStandardForm

\begin{itemize}
% p:augmentConstDecls

\item Some constants with names of form 
   $\mathit{c}\mathtt{\_base\_first}$
   or $\mathit{c}\mathtt{\_base\_last}$
   are used but not declared.  Victor adds declarations for such constants
   when they appear to be missing, when e.g. the constant
   $\mathit{c}\mathtt{\_first}$ is declared ($c$ not with suffix $\mathtt{\_base}$) and the constant 
   $\mathit{c}\mathtt{\_base\_first}$ is not declared.

\item The \fdl{} files output by the Examiner are missing declarations
  of the $\mathit{E}\_\_\mathtt{pos}$
  and $\mathit{E}\_\_\mathtt{val}$ functions used by each enumeration type
  $E$, including the implicitly declared \texttt{character} type.  These
  declarations are added in.

\item \fdl{} variables are considered as semantically the same as
  \fdl{} constants.  Each declaration of an \fdl{} variable $x$, is
  changed to a constant declaration, and new declarations are added
  for names \textit{x}\tttilde{} and \textit{x}\texttt{\%}.  \fdl{}
  units use the names \textit{x}\tttilde{} and \textit{x}\texttt{\%}
  to refer to the value of $x$ at procedure and loop starts
  respectively.


\item Occurrences of the \fdl{} operator \texttt{sqr(x)} are replaced by 
  \texttt{x ** 2}.

\item Distinct operators are introduced for the standard arithmetic operations
  $+$, $\times$, $-$(unary), $-$(binary) over the integers and reals,
  and an explicit coercion operator is introduced for converting integers
  to reals.

\item Distinct relations are introduced for the inequality relations 
  over integers, reals, and enumeration types. 

\item Distinct versions of the \fdl{} operator \texttt{abs(x)} are introduced
  for the real and integer types.   Defining axioms are added to the set
  of rules for each unit.

\item A defining axiom is added for the \fdl{} predicate \texttt{odd(x)}.

\item Some characterising axioms are added for the \fdl{} operator
   \texttt{bit\_\_or(x)}.  No axioms are added yet for other bit-wise
   arithmetic operators.

\item The \fdl{} language overloads the functions \texttt{succ} and
  \texttt{pred} and inequality relations such as $<$ and $\leq$.
  %
  Distinct versions are introduced for the \fdl{} \texttt{integer} type
  and each enumeration type and declarations are added for each of
  these versions.

\item The Examiner outputs rules with implicitly quantified variables.
  Victor infers the types of these variables and makes the
  quantifications explicit.  The explicit quantification is needed by
  all the provers to which Victor interfaces.

\end{itemize}


%----------------------------------------------------------------------------
\subsection{Type checking}
%----------------------------------------------------------------------------

Victor type checks units after translation into standard form
and after all translation steps have been applied.


%----------------------------------------------------------------------------
\subsection{Enumerated type abstraction}
%----------------------------------------------------------------------------

\begin{description}
\optionb{abstract-enums} 
  Replace enumerated types with abstract types, introduce all
  enumeration constants as normal constants, and keep all enumerated
  type axioms.  
  %
  These axioms are introduced by the Examiner to
  characterise enumerated-type-related functions such as
  $\mathit{E}\mathtt{\_\_val}$ and $\mathit{E}\mathtt{\_\_pos}$ and
  can serve as a partial axiomatisation of the introduced
  abstract types.

\optionb{elim-enums}      
  Replace each enumeration type $E$ with an integer subrange type $\{0
  \ldots k - 1 \}$ where $k$ is the number of enumeration constants in
  $E$.
  % 
  Declare each enumeration constant as a normal constant, and add an axiom
  giving its integer value.
  %
  Delete all existing enumerated type axioms and add in new axioms
  characterising enumerated-type-related functions such as 
  $\mathit{E}\mathtt{\_\_val}$ and  $\mathit{E}\mathtt{\_\_pos}$.

\optionb{axiomatise-enums}
  Replace each enumeration type $E$ with an uninterpreted type, and add
  axioms characterising the the uninterpreted type as isomorphic to the
  integer subrange $\{0 \ldots k - 1 \}$ where $k$ is the number of
  enumeration constants in $E$.
  %
  The added axioms replace the enumerated-type-related axioms
  introduced by the Examiner and provide a full axiomatisation of
  the enumerated types.

\end{description}

%----------------------------------------------------------------------------
\subsection{Early array and record abstraction}
%----------------------------------------------------------------------------

\begin{description}
\optionb{abstract-arrays-records-early} Enable abstraction at this point
\end{description}
See Section~\ref{sec:late-arr-rec-abs} below for rest of options

%----------------------------------------------------------------------------
\subsection{Separation of formulas and terms}
%----------------------------------------------------------------------------

In \fdl{} formulas are just terms of type Boolean.   Many provers require
the traditional first-order logic distinction between formulas and terms.
%
The options here control the introduction of this distinction.

Victor calls the term-level Booleans \emph{bits}.


\begin{description}
\optionb{bit-type} Enable separation.  
\optionb{bit-type-bool-eq-to-iff}
  Initially convert any equalities at Boolean type to `if and only if's.

\optionb{bit-type-with-ite} Whenever possible, introduce instances
  of the `if-then-else' operator rather than term-level versions of
  propositional logic operators and atomic relations.
  
\optionb{bit-type-prefer-bit-vals}
  A heuristic for controlling whether atomic relations are translated
  to term-level (bit-valued) functions or first-order-logic formula-valued
  relations.
  %
  With this heuristic, bit-valued functions are preferred.

\optionb{bit-type-prefer-props}
  Another heuristic for controlling whether atomic relations are translated
  to term-level (bit-valued) functions or first-order-logic formula-valued
  relations.
  With this heuristic, formula (propositional) relations are preferred.

  If neither this option or \texttt{-bit-type-prefer-bit-vals} is selected,
  the default behaviour is to use a bit-valued function just when 
  there is one or more occurrences at the term level.

\optionb{trace-prop-to-bit-insertion}
  Report in log file when a proposition-to-bit coercion (encoded using the
  `if then else' operator) is added.
 
\optionb{trace-intro-bit-ops-and-rels}
  Report in log file when term-level function is introduced for a function
  (either user-defined or built-in) that initially had Boolean value type.

\end{description}

NB: the \spark{} \fdl{} language has `bit' operators 
\texttt{bit\_\_or},
\texttt{bit\_\_and} and 
\texttt{bit\_\_xor}.
%
These \fdl{} operators take integers as arguments and return integers
as results.  Their result values correspond to the correct unsigned
binary result for the respective operations on unsigned binary
versions of the arguments.  Axioms on these operators capture the
arithmetic properties of Boolean operations on finite-length binary
words.
%
If the Victor option \texttt{-abstract-bit-ops} is used, Victor introduces 
operators
\texttt{bit\_\_\_or},
\texttt{bit\_\_\_and} and 
\texttt{bit\_\_\_xor}.
%
These operators work on the term-level Booleans introduced by Victor and
are distinct from the \spark{} \fdl{} bit operators.

%----------------------------------------------------------------------------
\subsection{Type refinement}
%----------------------------------------------------------------------------

\begin{description}
\optionb{refine-types} Master control

\optionb{refine-bit-eq-equiv} 
  Add in definition for bit-valued non-trivial equivalence relations.
  Needed when \\ \texttt{-bit-type-with-ite} option not previously selected.

\optionb{refine-int-subrange-type}       
\optionb{refine-bit-type-as-int-subtype}
\optionb{refine-bit-type-as-int-quotient}
\optionb{refine-array-types-with-quotient}

\optionb{refine-array-types-with-weak-extension-constraint}
  Constrain values of element and extended indices using possibly non-trivial
  equivalence relation on element type.  Default is to use equality to
  constrain these values.

  Only applies if option \texttt{-refine-array-types-with-quotient} is not
  selected.

\optionb{refine-uninterpreted-types}
  Refine every uninterpreted type to be predicate subtype of a new 
  uninterpreted type.  Use this to ensure that exists model in which 
  every uninterpreted type can be interpreted by some infinite set.

\optionb{no-subtyping-axioms}
  Suppress generation of axioms for sub-typing properties of functions and
  constants.

\optionb{no-functionality-axioms}
  Suppress generation of axioms for functionality properties of functions and
  relations.

\optionb{strong-subtyping-axioms}
  Use subtyping axioms without constraints on values of arguments.

\optionb{trace-refine-types-quant-relativisation}
  Report in log file whenever a quantifier is relativised.

\optionb{trace-refine-types-eq-refinement}
  Report in log file whenever an equality relation is refined to a non-trivial
  equivalence relation.

\optionb{trace-refine-types-bit-eq-refinement}
  Report in log file whenever an term-level equality relation is refined to 
  a non-trivial term-level equivalence relation.
\end{description}

%----------------------------------------------------------------------------
\subsection{Late array and record abstraction}
%----------------------------------------------------------------------------
\label{sec:late-arr-rec-abs}

\begin{description}
\optionb{abstract-arrays-records-late}  Enable abstraction at this point
  in translation.  

% Eliminate redundant operators
%------------------------------
\optionb{elim-array-constructors}
  Eliminate all occurrences of array constructors
\optionb{elim-record-constructors}
  Eliminate all occurrences of record constructors
\optionb{abstract-record-updates}
  Introduce axiomatic characterisations for record update operators in 
  terms of record constructors and record field selectors.

% Add axioms defining types axiomatically
%-----------------------------------------

\optionb{add-array-select-update-axioms}
  Assumes that array constructors have first been eliminated.

\optionb{add-array-extensionality-axioms}

\optionb{add-record-select-constructor-axioms}
  Assumes that record update operators have first been eliminated.

\optionb{add-record-constructor-extensionality-axioms}
  Add extensionality axioms involving record constructors and field select
  operators.

\optionb{add-record-select-update-axioms}
  Assumes that record constructors have first been eliminated.

\optionb{add-record-eq-elements-extensionality-axioms}
  Add extensionality axioms stating that records are equal just when
  all fields are equal.
  
\optionb{use-array-eq-aliases}
  Introduce aliases for equalities at array types in order
  to help with matching extensionality  axioms.

\optionb{use-record-eq-aliases}
  Introduce aliases for equalities at record types in order
  to help with matching extensionality  axioms.

% Abstract operators and types
%------------------------------
\optionb{abstract-array-select-updates}
  Change primitive array element select and update operators into 
  uninterpreted functions.
\optionb{abstract-array-types}
  Replace array types with uninterpreted types.


\optionb{abstract-record-selects-constructors}
  Change primitive record field selectors and constructors 
  into uninterpreted functions.
\optionb{abstract-record-selects-updates}
  Change primitive record field selectors and field update operators
  into uninterpreted functions.

\optionb{abstract-record-types}
  Replace record types with uninterpreted types.
\end{description}



%----------------------------------------------------------------------------
\subsection{Bit abstraction}
%----------------------------------------------------------------------------

\begin{description}
\optionb{abstract-bit-ops}
  Replace primitive bit-type operators with uninterpreted functions 
  and add characterising axioms

\optionb{abstract-bit-valued-eqs}
  Replace primitive bit-valued equality operators with uninterpreted functions 
  and add characterising axioms

\optionb{abstract-bit-valued-int-le}
  Replace primitive bit-valued integer inequality operators with
  uninterpreted functions and add characterising axioms

\optionb{elim-bit-type-and-consts}
  Replace primitive bit type with either integer type or $\subrange{0}{1}$ 
  subrange type, depending on whether type has been refined earlier or not.
  Replace primitive bit-type constants for true and false with $0$ and $1$.

%\optionb{abstract-bit-type-and-consts}
% Not implemented
\end{description}


%----------------------------------------------------------------------------
\subsection{Arithmetic simplification}
%----------------------------------------------------------------------------

\begin{description}
\optionb{elim-consts}
  %
  Eliminate integer constants. Rewrite all formulae using hypotheses of
  form $x = k$ or $x = -k$ where $x$ is an \fdl{} constant or variable,
  and $k$ is a natural number literal.  This eliminates the apparent
  syntactic non-linearity of some hypotheses and conclusions.  It is
  particularly useful for Yices which rejects formulae that appear
  non-linear.


\optionb{ground-eval-exp}
  Evaluate occurrences of exponent function with natural number arguments.

\optionb{ground-eval}
  Evaluate ground integer arithmetic expressions involving $+$, $-$
  (unary and binary), $\times$, integer division, integer modulus, and
  the exponent function.

\optionb{expand-exp-const}
  Expand natural-number powers of integer and real expressions into products,
  with special-case treatment for exponents 0 and 1.

\optionb{arith-eval}
  Apply the rewrite rules
  \begin{eqnarray*}
   k \times (k' \times e)   & =  &  kk' \times e \\
   (k \times e) \times k'   & =  &  kk' \times e \\
   e \times k               & =  &  k \times e \\
   (k \times e) \times (k' \times e') &  =  &  kk' \times (e \times e') \\
   e \times (k \times e')   &  = &  k \times (e \times e')  \\
   (k \times e) \times e'   & =  &  k \times (e \times e')  \\
  (k \times e) \div k'      & =  &   (k \div k') \times e  
      \quad\mbox{if $k'$ divides $k$}.
  \end{eqnarray*}
  The main aim of these rules is to eliminate instances of the $\div$
  operator.

\optionb{sym-consts}
  Replace each distinct natural number literal greater than 
  threshold \textit{t} with a new constant and assert axioms concerning
  how these new constants are ordered: if the new constants in increasing
  order are $c_1 \ldots c_n$, the axioms are 
  $t < c_1, c_1 < c_2, \ldots, c_{n-1} < c_n$.

  This option is used to try to reduce the frequency of machine
  arithmetic overflow with Simplify.  Other users of Simplify try
  thresholds of 100,000, though we've observed overflows with
  thresholds as low as 1000.

\optionv{sym-prefix}{prefix}
  Set prefix for new symbolic number constants.  Default prefix is
  \texttt{k\_\_\_}.


\end{description}



%----------------------------------------------------------------------------
\subsection{Arithmetic abstraction}
%----------------------------------------------------------------------------
The different interfaces and provers vary in the classes of arithmetic
operations they can handle.  These options allow one to abstract to
uninterpreted functions, possibly adding some characterising axioms,
when operations cannot be handled.

\begin{description}
%- - - - - - - - - - - - - - - - - - - - - - - - - - - - - - - - - - - 
\optionb{abstract-nonlin-times}
%- - - - - - - - - - - - - - - - - - - - - - - - - - - - - - - - - - - 
  Abstract each integer and real multiplication unless at least
  one of the arguments is a fixed integer or real constant.

  An \emph{integer constant} is a natural number $n$ or the expression $-n$.
  
  A \emph{real constant} is built from an integer constant using the
  \texttt{to-real} coercion, unary minus on the reals, and, optionally
  real division.  Real division is allowed just when the option 
  \texttt{-abstract-real-div} is not chosen.

  The \yices{} API usually rejects individual hypothesis or conclusion
  formulas if they have non-linear multiplications.  However, it does
  accept non-linear multiplications in quantified formulas, and
  will use linear instantiations of these formulas. 
  %
  Unfortunately, it currently aborts on finding a non-linear
  instantiation rather than simply rejecting the instantiation.

  The \smtlib{} sub-logics \textsc{auflia} and \textsc{auflira} both 
  require all multiplications to be linear.

%- - - - - - - - - - - - - - - - - - - - - - - - - - - - - - - - - - - 
\optionb{abstract-exp}
%- - - - - - - - - - - - - - - - - - - - - - - - - - - - - - - - - - - 
  Replace occurrences of integer and real exponent operators by new 
  uninterpreted functions.  Currently no defining axioms are supplied, though
  it would be easy to do so. 

  This abstraction only happens after possibly evaluating ground 
  and constant exponent instances.

  Only the \cvcthree{}-via-API prover alternative can handle these
  operators directly.

%- - - - - - - - - - - - - - - - - - - - - - - - - - - - - - - - - - - 
\optionb{abstract-divmod}
%- - - - - - - - - - - - - - - - - - - - - - - - - - - - - - - - - - - 
  Replace occurrences of integer division and modulus operators by new 
  uninterpreted functions.  

%   There are two alternative rule sets provided
%   that can be loaded using the \texttt{-rules} option.
%   \begin{itemize}
%   \item \texttt{divmod.rul}: Exactly characterises \texttt{div} and gives
%     bounds on \texttt{mod} values.
%   \item \texttt{divmod-full.rul}
%     Exactly characterises both \texttt{div} and \texttt{mod}. 
%     These rules are probably more than is necessary in most cases. 
%     %
%     They have not been tried recently.  
%   \end{itemize}

%- - - - - - - - - - - - - - - - - - - - - - - - - - - - - - - - - - - 
\optionb{abstract-real-div}
%- - - - - - - - - - - - - - - - - - - - - - - - - - - - - - - - - - - 
  Abstract occurrences of real division to a new uninterpreted function.
  No characterising axioms are currently provided.

  \yices-API, \cvcthree-API and \zthree-\smtlib{} all allow input 
  with the real division operator, though it is not known what kinds of
  occurrences are accepted in each case.

  The official \smtlib{} logics involving reals do not allow real division.
  The assumption is that pre-processing has eliminated all occurrences of
  real division.  Victor doesn't yet carry out such pre-processing.

%- - - - - - - - - - - - - - - - - - - - - - - - - - - - - - - - - - - 
\optionb{abstract-reals}
%- - - - - - - - - - - - - - - - - - - - - - - - - - - - - - - - - - - 
  Abstract occurrences of real arithmetic operations 
  ($+$, unary $-$, binary $-$, $*$, $/$), integer to real coercions, and
  real inequalities to new uninterpreted functions.

  Currently this is needed by the \smtlib{} and Simplify translations.
  The \smtlib{} driver does not attempt to make use of the limited
  support in some of the \smtlib{} sub-logics for reals.

  This option is not necessary when \cvcthree{} and \yices{} are
  invoked via their APIs, as both APIs support real arithmetic.

\end{description}




%----------------------------------------------------------------------------
\subsection{Final translation steps}
%----------------------------------------------------------------------------

\begin{description}
\optionb{elim-type-aliases}
  Normalise all occurrences of type identifiers in type,
  constant, function and relation declarations and in all formulas.
  %
  Normalisation eliminates all occurrences of type ids $T$ that have a
  definition $T \doteq T'$ where $T'$ is either a primitive atomic
  type (Boolean, integer, integer subrange, real or bit type) or is
  itself a type id.

  This is needed for the \smtlib{} and Simplify translations.

\optionb{switch-types-to-int}
  Replace all occurrences of type identifiers in 
  constant, function and relation declarations and in all formulas with
  the integer type. 
  % 
  Checks that every defined type is either an alias for another defined
  type or an alias for the integer type. 
  %
  This translation step assumes that a countably infinite model exists
  for every uninterpreted type. 

  This option is is needed for the Simplify translation.

  
\optionb{lift-quants}
  Apply the rewrite rule
  \[
      P \Implies \forall x:T.\ Q \quad\Iff\quad  \forall x:T.\ P \Implies Q
  \]
  ($x$ not free in $P$) 
  to all formulae.  The quantifier instantiation heuristics in both 
  \zthree{} and Simplify work better when universal quantifiers in 
  hypotheses are all outermost.

\optionb{strip-quantifier-patterns}

Some of the universally-quantified axioms introduced by translation
have trigger patterns giving hints on how instantiations can be guessed.
%
This option strips out these patterns.

\end{description}



%============================================================================
\section{CSV utilities}
%============================================================================
These utilities are very useful for analysing and comparing results of Victor
runs.

%----------------------------------------------------------------------------
\subsection{Filter CSV records}
%----------------------------------------------------------------------------
Usage:
\begin{quote}
   \texttt{csvfilt} [\texttt{-v}] \textit{n str} [\textit{file}]
\end{quote}
Filter \csv{} records, returning on standard output just those with
\textit{str} a substring of field \textit{n} (1-based).
%
If \texttt{-v} is provided, then returns those records without
\textit{str} a substring of field \textit{n}.
Records are drawn from file \textit{file} if it is supplied. If not, they
are taken from standard input.

%----------------------------------------------------------------------------
\subsection{Merge two CSV files}
%----------------------------------------------------------------------------
Usage:
\begin{quote}
   \texttt{csvmerge} \textit{file1 m1 \ldots{} mj file2 n1 \ldots{} nk}
\end{quote}
The files \textit{file1} and \textit{file2} must have the same number
of records.
This command merges corresponding records from the two files and outputs
them on standard output.
The merged records are composed from 
fields \textit{m1 \ldots{} mj} in the records in \textit{file1} and 
fields \textit{n1 \ldots{} nk} in the records in \textit{file2}.
If \textit{j} = 0, all fields of  \textit{file1} records are used.
If \textit{k} = 0, all fields of  \textit{file2} records are used.
Fields are numbered starting from 1.

%----------------------------------------------------------------------------
\subsection{Project out fields of CSV records}
%----------------------------------------------------------------------------
Usage:
\begin{quote}
  \texttt{csvproj} [\texttt{-v}] \textit{n1 \ldots{} nk} [\textit{file}]
\end{quote}
Build new records from fields \textit{n1 \ldots{} nk} of the input
records and output to standard output.  Input records are drawn from
file \textit{file} if it is supplied. If not, they are taken from
standard input.  If option \texttt{-v} is supplied, then all fields
but \textit{n1 \ldots{} nk} are used to build the output records.
Fields are numbered starting from 1.

%----------------------------------------------------------------------------
%\subsection{Intersecting CSV files}
%----------------------------------------------------------------------------
Usage:
\begin{quote}
  \texttt{csvisect} \textit{file1} \textit{file2}
\end{quote}
Print on standard output those records that occur in both
\textit{file1} and \textit{file2}.  Comparison of records currently
just uses string equality, so it is sensitive to whitespace between
record fields. 


%============================================================================
%\section{Known limitations}
%============================================================================
% \begin{enumerate}
% \item 
% \end{enumerate}

%============================================================================
%\section{Known problems}
%============================================================================

% \begin{enumerate}
% \item 
% \end{enumerate}

%============================================================================
\section{Missing Documentation}
%============================================================================
The current Victor has several features that are not properly documented
yet in this manual.  These include:
\begin{itemize}
\item Support for outputing VCs for proof using the Isabelle/HOL
  theorem prover.

  The current release includes some preliminary code for this.
  Improved code has been developed and is waiting to be merged in.
\end{itemize}


%============================================================================
%\section{Future developments}
%============================================================================

% Anticipated by end of 2010 are
% \begin{itemize}
% \item Support for outputing VCs for proof using the Isabelle/HOL
%   theorem prover.

%   The current release includes some preliminary code for this.
%   Improved code has been developed and is waiting to be merged in.

% \item A fully-working Windows port.

%   The current release has some code for this, but it is not yet fully tested.
% \end{itemize}


% Items in the pipeline include
% \begin{enumerate}
% \cbstart
% \item Improving support in \smtlib{} translation for reals.
% \item API interface for \zthree.
% \cbend
% \item Output of VCs in format for reading into interactive 
%   theorem provers such as  PVS and HOL Light.
% \item Exploiting bit-vector handling capabilities of \smt{} solvers.
% \end{enumerate}


\end{document}

%%% Local Variables: 
%%% mode: latex
%%% TeX-master: t
%%% End: 
